\chapter{Related Work} \label{chap:Chapter2}       
\epigraph{``This is where technology is now, imagine where we can go in the future” }{\textit{Timothy Chung}}

Σε αυτό το \emph{Chapter} περιγράφονται τρόποι - από την βιβλιογραφία - με τους οποί\-ους, υπάρχουσες εφαρμογές 
μεμονωμένων \hyperref[abbr:UAV]{UAV}, καθώς και drone swarms επιλύουν το localization pro\-blem. Κάποια από τα συστήματα στις παρακάτω έρευνες έχουν καθαρά
θεωρητική πλευρά, ενώ άλλα έχουν δοκιμαστεί σε real-life scenarios.
Τέλος, σε αρκετές από αυτές τις εφαρμογές χρησιμοποιούνται τεχνικές που θα αναλυθούν στο \emph{Chapter} \ref{chap:Chapter3} 
σε μεγαλύτερο βάθος, καθώς θα αποτελέσουν το προαπαιτούμενο θεωρητικό και μαθηματικό υπόβαθρο για τον σχεδιασμό του συστήματος της παρούσας διπλωματικής.

Κυρίως, αυτά που θα μελετήσουμε θα αφορούν Cooperative Localization (\hyperref[abbr:CL]{CL}) των \hyperref[abbr:UAV]{UAV}s, ενώ επίσης
μας ενδιαφέρουν τα Distributed Computation - Real Time Location System (\hyperref[abbr:RTLS]{RTLS}) \cite{rtls}. Το πρόβλημα του Local Positioning System 
(\hyperref[abbr:LPS]{LPS}) \cite{lps} έχει ερευνηθεί από διάφορες οπτικές\udot οι οποίες μπορεί να βασίζονται σε 
\hyperref[abbr:RF]{RF}, sound waves ή ακόμα και να είναι image oriented.

Σχεδόν βέβαιο είναι επίσης, ότι από όποια κατεύθυνση και αν προσεγγίσουμε το \hyperref[abbr:LPS]{LPS} να έχουμε την ανάγκη να 
χρησιμοποιήσουμε κάποια τεχνική για να συνενώσουμε όλες τις πληροφορίες που έχουμε από τους διάφορους αισθητήρες για κάθε μεμονωμένο 
time-frame. Την ανάγκη αυτού έρχεται να καλύψει η έννοια του \emph{sensor fusion} \cite{sensor-fusion}, με μερικούς γνωστούς 
αλγόριθμους που να το επιτυγχάνουν - να είναι το  
Extended Kalman Filter (\hyperref[abbr:EKF]{EKF}), Unscented Kalman Filter (\hyperref[abbr:UKF]{UKF}), Covariance Intersection  
Filter (\hyperref[abbr:CIF]{CIF}),  Split  Covariance  Intersection  Filter (\hyperref[abbr:SCIF]{SCIF}) και  Belief  Propagation 
(\hyperref[abbr:BP]{BP}) \cite{fusion-filters}. 

% Τέλος, είναι επίσης σημαντικό να αναφερθούν τα βασικά \emph{coordinate frames} τα οποία χρησιμοποιούνται. 
% Αισθητήρες όπως το \hyperref[abbr:IMU]{IMU} και το compass κάνουνε μετρήσεις με γνώμονα το ίδιο το σώμα 
% του αντικειμένου για αυτό - σε αυτήν την περίπτωση το σύστημα αξόνων ονομάζεται body frame (B frame) \cite{uwb-imu-gps1}.
% Συχνά είναι όμως βολικό να μετατρέψουμε αυτές τις μετρήσεις σε ένα κοινό North East Down (\hyperref[abbr:NED]{NED}) 
% σύστημα, οπότε τότε αναφερόμαστε σε N frame \cite{uwb-imu-gps1} \cite{body-frames}. Παράδειγμα μεταξύ των δύο 
% συστημάτων βρίσκεται στο \emph{Figure} \ref{fig:Important-coordinate-frames}.

% \begin{figure} [H]
% 	\centering
% 	\includegraphics[width=0.6\linewidth]{Images/Related-Work/b-frame-n-frame-and-Euler-angles.png}
% 	\decoRule
% 	\caption[Important coordinate frames]{Important coordinate frames \cite{body-frames}}
% 	\label{fig:Important-coordinate-frames}
% \end{figure}


% ------------------------------------------------------------------------------------
\section{Radio Frequency}
Το πιο εύκολο που μπορούμε να φανταστούμε είναι να χρησιμοποιήσουμε \hyperref[abbr:RF]{RF} τε\-χνο\-λο\-γίες
όπως WiFi, Zigbee και Ultra-Wideband (\hyperref[abbr:UWB]{UWB}) για την επικοινωνία και την υλοποίηση του \hyperref[abbr:LPS]{LPS}.   
Το WiFi είναι εύκολα προσβάσιμο\udot με μικρό κόστος ενώ το Zigbee προτιμάται για
low power consumption εφαρμογές. Όμως προβλήματα αυτών των δύο είναι, ότι το 
WiFi έχει μικρή εμβέλεια\footnote{Αυτό είναι κυρίως πρόβλημα για outdoor scenarios} όπως επίσης είναι πολύ εύκολο 
να υπάρχουν παρεμβολές από τις υπόλοιπες γειτονικές συσκευές\footnote{Αυτό είναι κυρίως πρόβλημα για indoor scenarios} 
άρα δεν το καθιστά καλή λύση για swarms. Το Zigbee επιλύει κάποια από αυτά τα 
προβλήματα\footnote{Σε indoor σενάρια παραμένει η δυσκολία για collision-free formation}. Συχνότερη εφαρμογή έχουν 
τα \hyperref[abbr:UWB]{UWB} καθώς επιλύει θέματα σχετικά με την ακρίβεια των μετρήσεων, το κόστος καθώς και την 
εμβέλεια\udot ταυτόχρονα \cite{uwb-imu-gps3}. Τέλος, με την χρήση του \hyperref[abbr:UWB]{UWB} επίσης, μπορούμε εύκολα να δημιουργήσουμε
ένα Flying Ad-hoc Network (\hyperref[abbr:FANET]{FANET}) το οποίο βοηθάει στην ευελιξία του συστήματος.


% ---------------------


\subsection{UWB, IMU and GPS}
Στο \cite{uwb-imu-gps1} θεωρείται ότι υπάρχουν Ν αριθμημένα \hyperref[abbr:UAV]{UAV}, τα οποία επικοινωνούν σε \hyperref[abbr:UWB]{UWB}, 
αξιοποιώντας την λογική του γράφου - για την αναπαράσταση τους - όπως 
περιγράφτηκε στο \emph{Section} \ref{sec:Chapter3-3}. Ενώ κάθε node περιλαμβάνει \hyperref[abbr:IMU]{IMU}, Compass, \hyperref[abbr:GPS]{GPS} και A\-lti\-tu\-de and Heading Re\-fe\-re\-nce 
System (\hyperref[abbr:AHRS]{AHRS})\footnote{Αξιοποιείται από ένα flight control unit} με την χρήση των οποίων μπορούν σε N frame να συλλέξουν 
πληροφορίες όπως παρουσιάζονται στους πα\-ρα\-κά\-τω πίνακες για κάθε time-frame k.

\begin{gather*}
	\textbf{GPS Position:}\quad r^k_{i, GPS} = \left[x^k_{i, GPS} \quad y^k_{i, GPS} \quad z^k_{i, GPS}\right]^T \\
	\textbf{Velocity estimation:}\quad\hat{v}^k_i = \left[\hat{v}^k_{xi} \quad \hat{v}^k_{yi} \quad \hat{v}^k_{zi}\right]^T \\
	\textbf{Acceleration estimation:}\quad\hat{a}^k_i = \left[\hat{a}^k_{xi} \quad \hat{a}^k_{yi} \quad \hat{a}^k_{zi}\right]^T
\end{gather*}

Σε αυτό το \hyperref[abbr:UWB]{UWB} δίκτυο είναι εύκολο το κάθε drone να μετρήσει την απόσταση του από ένα γειτονικό
μέσω \hyperref[abbr:ToA]{ToA}-based \hyperref[abbr:RTT]{RTT} μεθόδου, με όνομα Single-sided Two-way Ranging (\hyperref[abbr:SS-TWR]{SS-TWR}),
όπως περιγράφει η εξίσωση (\ref{eq:ss-twr}). Τα $T_{round}$ και $T_{reply}$ είναι
οι χρόνοι που μετρήθηκαν για την εκτίμηση απόστασης - η οποία όμως περιλαμβάνει σφάλμα ως εξής, $\norm{r_i - r_j}$ η πραγματική απόσταση και $n_{UWB}$ τυχαία μεταβλητή με τυπική απόκλιση ίση με 
$0.1m$\footnote{Βάση θεώρησης των συγγραφών της έρευνας} και στο \emph{Figure} \ref{fig:SS-TWR} υπάρχει παράδειγμα αυτής.

\begin{gather}
    d_{ij} = \frac{1}{2}(T_{round} - T_{reply}) \times c = \norm{r_i - r_j} + n_{UWB}, \quad with \quad n_{UWB} \sim N(0, \sigma^2_{UWB}) \label{eq:ss-twr}
\end{gather}

\begin{figure} [H]
	\centering
	\includegraphics[width=0.69
	\linewidth]{Images/Related-Work/Single-Sided-Two-Way-Ranging-SS-TWR-5.png}
	\decoRule
	\caption[Illustration of Single-sided Two-way ranging]{Illustration of Single-sided Two-way ranging \cite{uwb-imu-gps1}}
	\label{fig:SS-TWR}
\end{figure}

Παρόμοιας λογικής σφάλματα θεωρείται ότι συμπεριλαμβάνουν και οι μετρήσεις από
τους άλλους αισθητήρες, ενώ επίσης ότι το κάθε node μπορεί να μεταφέρει τις πληροφορίες 
αυτές στα υπόλοιπα.

Πριν το relative position estimation, για λόγους που έχουν να κάνουν με 
Ranging outlier rejection και Ranging failure data completion γίνεται ένα 
preprocessing με χρήση του \hyperref[abbr:EKF]{EKF} στα δεδομένα - όπως 
παρουσιάζεται στο \emph{Figure} \ref{fig:Data-preprocessing-work-flow-using-EKF} -
ώστε να έχουμε σαν αποτέλεσμα μετά από αυτό το στάδιο μία καλύτερη εκτίμηση των αποστάσεων $d_{ij}$ 
για κάθε \emph{i} και \emph{j} nodes. 

\begin{figure} [H]
	\centering
	\includegraphics[width=\linewidth]{Images/Related-Work/Data-preprocessing-work-flow-using-EKF.png}
	\decoRule
	\caption[Data preprocessing work flow using EKF]{Data preprocessing work flow using EKF\cite{uwb-imu-gps1}}
	\label{fig:Data-preprocessing-work-flow-using-EKF}
\end{figure}

Ενώ σε επόμενο στάδιο, για το κομμάτι του relative position estimation, θεωρούν το σύστημα ως ένα γράφο 
με λογική ενός \emph{spring system} - με διαφορετικούς $k_o$ συ\-ντε\-λεστές σκληρότητας - όπου στην κατάσταση 
όπου το σύστημα σταθεροποιηθεί, θεωρείται ότι είναι το σημείο
το οποίο περιγράφει την τοποθεσία των nodes. Ψάχνουν να βρουν το σημείο στο οποίο η συνολική δυναμική ενέργεια 
στο σύστημα είναι η μικρότερη. Ουσιαστικά μετατρέπουν το position estimation πρόβλημα σε ένα 
non-convex optimization πρόβλημα και προσπαθούν να το λύσουν. Ένα overview του συστήματος που δημιούργησαν 
παρουσιάζεται στο \emph{Figure} \ref{fig:paper1-overview}.

\begin{figure} [H]
	\centering
	\includegraphics[width=0.69
	\linewidth]{Images/Related-Work/UAV-swarm-system-diagram.png}
	\decoRule
	\caption[System overview]{System overview for paper \cite{uwb-imu-gps1}}
	\label{fig:paper1-overview}
\end{figure}


% ----------------------------------------------------------------





