\begin{abstract}
	\renewcommand*\abstractname{Περίληψη}
	\addchaptertocentry{\abstractname} % Add the abstract to the table of contents
	Η ανάγκη προσδιορισμού της θέση αντικειμένων έχει παρουσιαστεί εδώ και χι\-λιά\-δες χρόνια, με τα τελευταία 70 να αξιοποιούμε ακόμα και ψηφιακά μέσα. 
	Ταυτόχρονα, μία από τις πλέον αναπτυσσόμενες τεχνολογίες είναι αυτή των UAV. 
	Συνδυάζοντας τις δύο αυτές ιδέες, στην συγκεκριμένη εργασία γίνεται προσπάθεια να υλοποιηθεί η πρώτη γενιά ενός οικονομικού συστήματος εκτίμησης του position αντικειμένου από συνεργατικά σμήνη drones. 
	Η διεξαγωγή ενός prior art search, βοήθησε στο να γίνει διακριτό το πλήθος των διαφορετικών μεθόδων που ήδη χρησιμοποιούνται στην βιβλιογραφία για επίλυση παρόμοιων προβλημάτων. 
	Συχνά επιλέγεται η χρήση ραδιοσυχνοτήτων, όπως GPS, WiFi, UWB, κυψελωτά δίκτυα και δίκτυα Lora ώστε να πραγματοποιηθεί το localization, άλλες φορές ηχητικά κύματα, επιπλέον, η χρήση καμερών και LiDAR αποτελούν αρκετά δημοφιλής προσέγγισης. 
	Ενώ, ήταν σημαντικό να γίνουν κατανοητοί οι μαθηματικοί φορμαλισμοί που ακολουθούνται σε κάθε προσέγγιση, ώστε να καθοριστεί η πολυπλοκότητας τους, διότι με βάση τις αρχές λειτουργίας τους και τους πόρους που χρειάζεται ο καθένας, καταλήγουμε να επιλέγεται και το hardware του ενσωματωμένου συστήματος. 
	Αυτές αναφέρονται στον τρόπο υπολογισμού ενδιάμεσων πληροφοριών - όπως εκτίμηση απόστασης και γωνίας - μέσω τεχνικών RSSI, TDoA, AoA ακόμα και Stereo Vision, οι οποίες χρησιμοποιούνται σε αλγόριθμους όπως τον Trilateration, Triangulation ή Hyperbolic Positioning για την εκτίμηση τελικά της θέσης. 
	Με την γνώση των αναγκών της κάθε προσέγγισης, έγινε η σχεδίαση ενός συστήματος δύο επιπέδων, όπου μέσω monocular vision και structure from reference γίνεται εκτίμηση της απόστασης καθορισμένου αντικειμένου από worker nodes, και μέσω της αρχής λειτουργίας του Μultilateration, επιτεύχθηκε στο master node να υπολογιστεί η θέση του object. 
	Πειράματα πραγματοποιήθηκαν σε εσωτερικό και εξωτερικό χώρο από μεμονωμένο κόμβο, ενώ προσομοιώθηκε και η λειτουργία στο σύνολο του συστήματος. Για λόγους πληρότητας/επαλήθευσης ανίχνευσης, υλοποιήθηκε επίσης μηχανισμός προσδιορισμού του ID του εκτιμώμενου αντικειμένου μέσω ανάλυσης συχνότητας ενός blinking led.
\end{abstract}