\begin{abstract}
	\renewcommand*\abstractname{Περίληψη}
	\addchaptertocentry{\abstractname} % Add the abstract to the table of contents
	Η ανάγκη προσδιορισμού της θέση αντικειμένων έχει παρουσιαστεί εδώ και χι\-λιά\-δες χρόνια, με τα τελευταία 70 να αξιοποιούμε ακόμα και ψηφιακά μέσα για να την επιτύχουμε.  
	Συνδυάζοντας την ιδέα αυτή με τη πλέον αναπτυσσόμενη τεχνολογία των UAV, στην συγκεκριμένη εργασία γίνεται προσπάθεια υλοποίησης της πρώτης γενιάς ενός οικονομικού συστήματος εκτίμησης του position αντικειμένου από συ\-νε\-ργα\-τι\-κά σμήνη drones. 
	Για να γίνει διακριτό το πλήθος των διαφορετικών μεθόδων που ήδη χρησιμοποιούνται στην βι\-βλιο\-γρα\-φία για επίλυση παρόμοιων προβλημάτων
	διεξήχθη ένα prior art search.
	Συχνά επιλέγεται η χρήση ραδιοσυχνοτήτων, όπως GPS, WiFi, UWB, κυψελωτά δίκτυα και δίκτυα Lora ώστε να πραγματοποιηθεί το localization, άλλες φορές ηχητικά κύματα, και άλλες οπτικά, όπως με χρήση καμερών και LiDAR. 
	Ακόμη, ήταν ση\-μα\-ντι\-κό να γίνουν κατανοητοί οι μαθηματικοί φο\-ρμα\-λι\-σμοί που ακολουθούνται σε κάθε προσέγγιση, ώστε να καθοριστεί η πολυπλοκότητας τους. Αφού, βάση των αρχών λειτουργίας τους και των πόρων που χρειάζεται ο καθένας, καταλήγουμε στο κατάλληλο hardware του ενσωματωμένου συστήματος. 
	Αυτές αφορούν στον τρόπο υπολογισμού ενδιάμεσων πληροφοριών - όπως εκτίμηση απόστασης και γωνίας - μέσω τεχνικών RSSI, TDoA, AoA ακόμα και Stereo Vision, που χρησιμοποιούνται σε αλγόριθμους όπως τον Trilateration, Triangulation ή Hyperbolic Positioning για την εκτίμηση τελικά της θέσης. 
	Με την γνώση των αναγκών της κάθε προσέγγισης, έγινε η σχεδίαση ενός συστήματος δύο επιπέδων, όπου μέσω monocular vision και structure from reference εκτιμήθηκε η απόσταση καθορισμένου αντικειμένου από worker nodes, και μέσω της αρχής λειτουργίας του Multilateration, επιτεύχθηκε στο master node να υπολογιστεί η θέση του object. 
	Πειράματα πραγματοποιήθηκαν σε εσωτερικό και εξωτερικό χώρο από μεμονωμένο κόμβο, ενώ προσομοιώθηκε και η λειτουργία στο σύνολο του συστήματος. Για λόγους πληρότητας/επαλήθευσης ανίχνευσης, υλοποιήθηκε επίσης μηχανισμός προ\-σδιο\-ρι\-σμού του ID του εκτιμώμενου αντικειμένου μέσω ανάλυσης συχνότητας ενός blinking led.
\end{abstract}