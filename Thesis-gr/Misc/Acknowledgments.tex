\begin{acknowledgements}
	\renewcommand\acknowledgementname{Ευχαριστίες}
	\addchaptertocentry{\acknowledgementname} % Add the acknowledgements to the table of contents
	
	Η διεξαγωγή μίας διπλωματικής εργασίας δεν είναι ατομική αρμοδιότητα. Πολλά άτομα υπάρχουν με τον ένα ή άλλο τρόπο πίσω από αυτήν, χωρίς την βοήθεια των οποίων δεν θα είχε ολοκληρωθεί.

	Αρχικά, είμαι σίγουρος ότι η συγκεκριμένη εργασία δεν θα ήταν εφικτή χωρίς την βοήθεια του επιβλέπων καθηγητή μου. Θα ήθελα συνεπώς να ευχαριστήσω τον καθηγητή και επιβλέπων της συγκεκριμένης διπλωματικής εργασίας, Καθ. Α\-πό\-στο\-λο Δόλλα, ο οποίος μου έδωσε την ευκαιρία να πραγματοποιήσω την εργασία μου υπό την καθοδήγηση του, και ήταν δίπλα μου σε όλη την διάρκεια της, όχι μόνο με συμβουλευτική, καθοδηγητική και τεχνική φύση αλλά και με ανθρώπινη. Ένας επίσης σημαντικός - για εμένα - λόγος, για τον οποίο θα ήθελα να ευχαριστήσω τον κύριο Δόλλα, είναι διότι, ήταν από τους καθηγητές που κατά την διάρκεια των σπουδών μου\udot υπήρξε μία καθοριστική παρουσία καθηγητή και ανθρώπου, που μέσω του τρόπου διδασκαλίας του, και του πάθος προς την επιστήμη του μηχανικού, με έκανε αγαπήσω τον χώρο των ενσωματωμένων συστημάτων τον οποίο μέχρι τότε δεν γνώριζα, και τελικά ωθήθηκα να ασχοληθώ με αυτόν.

	Επίσης, θα ήθελα να ευχαριστήσω τους καθηγητές, Αναπ. Καθ. Ευτύχιο Κουτρούλη και Αναπ. Καθ. Παναγιώτη Παρτσινέβελο, ως μέλη της τριμελής επιτροπής, καθώς, και για την βοήθεια τους, στον προσδιορισμό των requirements που έπρεπε να ικανοποιεί το σύστημα.  

	Συμπληρωματικά, θα ήθελα να ευχαριστήσω το Πολυτεχνείο Κρήτης και όλους τους υπόλοιπους καθηγητές για τις γνώσεις που μου έδωσαν όλα αυτά τα χρόνια, οι οποίες με βοήθησαν κατά την διάρκεια της διπλωματικής μου. Σημαντική ήταν επίσης η βοήθεια των εργαστηρίων MHL και SenseLab, μαζί με το διδακτικό προσωπικό και τους συναδέλφους που βρίσκονται σε αυτά. Θα ήθελα συνεπώς να ευχαριστήσω τον κύριο Κιμιωνή Μάρκo καθώς και τους συναδέλφους που με βοήθησαν να σχεδιάσω το σύστημα και να ξεπεράσω προβλήματα που προέκυψαν στην υλοποίηση του, συ\-γκε\-κρι\-μέ\-να τους, Αντωνόπουλο Άγγελο, Φελέκη Παναγιώτη και Φωτάκη Τζανή. 

	Τέλος, Θα ήθελα να ευχαριστήσω τους κρυφούς ήρωες μου. Τους ανθρώπους που ήταν εκεί, ήδη από τα πρώτα βήματα, πριν ακόμα ξεκινήσω το ταξίδι στο κόσμο του μηχανικού. Ένα μεγάλο ευχαριστώ συνεπώς ανήκει στους γονείς, συγγενείς και φίλους που ήταν δίπλα μου όλα αυτά τα χρόνια, καθώς και στις Άννα Μαρία Ζα\-μπε\-τά\-κη και Αγγελική Σηφάκη για την υποστήριξη τους. 

	\begin{flushright}Σπυριδάκης Χρήστος\\ Χανιά, 2022\end{flushright}
\end{acknowledgements}
